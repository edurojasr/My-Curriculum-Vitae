%-------------------------------------------------------------------------------
%	SECTION TITLE
%-------------------------------------------------------------------------------
\cvsection{Programas y Certificaciones}


%-------------------------------------------------------------------------------
%	CONTENT
%-------------------------------------------------------------------------------
\begin{cventries}


%---------------------------------------------------------
  \cventry
  {Getting started with Python} % Affiliation/role
    {LinkedIn} % Organization/group
    {LinkedIn Learning} % Location
    {2023} % Date(s)
    {
      \begin{cvitems} % Description(s) of experience/contributions/knowledge
        \item Datas structures.
        \item Software Development.
        \item Python.
      \end{cvitems}
    }
%---------------------------------------------------------
  \cventry
  {Análisis de imágenes satelitales con QGIS} % Affiliation/role
    {SOLTIG} % Organization/group
    {Virtual} % Location
    {08 de agosto al 23 de septiembre del 2022} % Date(s)
    {
      \begin{cvitems} % Description(s) of experience/contributions/knowledge
        \item Descarga y preprocesamiento de imágenes satelitales con QGIS.
        \item Clasificación y manejo de imágenes satelitales.
        \item Utilización de OrfeoTools para el procesamiento de imágenes satelitales.
      \end{cvitems}
    }
%---------------------------------------------------------
  \cventry
    {Certificación en Aprendizaje Automático Machine Learning} % Affiliation/role
    {Texas Tech University} % Organization/group
    {Texas Tech University - Costa Rica} % Location
    {2021} % Date(s)
    {
      \begin{cvitems} % Description(s) of experience/contributions/knowledge
        \item Machine Learning Fundamentals: Python and Scikit-Learn.
        \item Deep Learning: Deep Neural Networks with TensorFlow and Keras.
        \item Natural Language Processing: TensorFlow and NLTK.
        \item Unsupervised Machine Learning: Scikit-learn and TensorFlow.
        \item Reinforcement Learning: TensorFlow, OpenAI GYM and TF-Agents.
      \end{cvitems}
    }
%---------------------------------------------------------
  \cventry
  {Analizando y Modelando con QGIS} % Affiliation/role
    {SOLTIG} % Organization/group
    {Virtual} % Location
    {05 de septiembre al 03 de octubre del 2021} % Date(s)
    {
      \begin{cvitems} % Description(s) of experience/contributions/knowledge
        \item Modelado utilizando QGIS, GRASS y SAGA.
        \item Generación de Modelos de Elevación, cuencas y escorrentía.
        \item Procesado por lotes y publicación web.
      \end{cvitems}
    }
%---------------------------------------------------------
  \cventry
  {QGIS en la Gestión Ambiental} % Affiliation/role
    {SOLTIG} % Organization/group
    {Virtual} % Location
    {07 de septiembre al 03 de diciembre del 2020} % Date(s)
    {
      \begin{cvitems} % Description(s) of experience/contributions/knowledge
        \item Modelos de interpolación.
        \item Cálculo del Índice de Vegetación corregido.
        \item Análisis de escorrentía y drenajes.
      \end{cvitems}
    }
%---------------------------------------------------------
  \cventry
  {Gestor de Innovación} % Affiliation/role
    {Programa de Formación Práctica en Innovación Orientada al
    Mercado} % Organization/group
    {UTN Costa Rica - Universidad de Leipzig Alemania} % Location
    {2020} % Date(s)
    {
      \begin{cvitems} % Description(s) of experience/contributions/knowledge
        \item Programa de especialización en Innovación enfocada al
        Mercado.
        \item Capacitación en el liderazgo de Proyectos y Procesos de
        Innovación.
      \end{cvitems}
    }
%---------------------------------------------------------

\end{cventries}
